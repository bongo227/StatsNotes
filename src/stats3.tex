\cchapter{Statistics 3}

\section{Contingency Tables}

    \subsection{Contingency Tables}
        So far we have looked at the frequency a single event happens, but sometimes your interested in the frequencies of two criteria happening.
        
        \begin{example}
        {
            Trains from 3 stations leave late or on-time, test at 5\% is their is any association.\\
        
            \begin{center}
            \begin{tabular}{c|c|c}
                  & On Time & Late \\
                \hline
                A & 26      & 14   \\
                B & 30      & 10   \\
                C & 44      & 26   \\
            \end{tabular}
            \end{center}
        }
        
        
        \begin{step}{Hypothesis}
        $H_0$: No association
        
        $H_1$: Association
        \end{step}
        
        \begin{step}{Calculate the expected and $\chi^2$ values}
        \begin{center}
        \begin{tabular}{c|c|c}
            $O_i$ & $E_i$           & $\displaystyle\frac{(O_i - E_i)^2}{E_i}$   \\[2ex]
            \hline
            \rule{0pt}{3.5ex} 
            26    & $\frac{80}{3}$  & $\frac{1}{60}$                \\[1ex]
            30    & $\frac{80}{3}$  & $\frac{5}{12}$                \\[1ex]
            44    & $\frac{140}{3}$ & $\frac{1}{60}$                \\[1ex]
            14    & $\frac{40}{3}$  & $0.152$                       \\[1ex]
            10    & $\frac{40}{3}$  & $\frac{1}{30}$                \\[1ex]
            16    & $\frac{70}{3}$  & $0.305$                       \\[1ex]
        \end{tabular}
        \end{center}
        \end{step}
    
        \begin{step}{Find the statistic $\chi^2$}
        $$\sum{\frac{(O_i - E_i)^2}{E_i}} = \chi^2 = 1.757$$
        \end{step}
        
        \begin{step}{Find the critical value}
        $$v = (3-1)\times(2-1) = 2$$
        $$\chi^2_2(0.95) = 5.991$$
        \end{step}
        
        \begin{step}{Conclusion}
        $1.757 < 5.991$ so accept $H_0$, evidence suggests stations and lateness are not associated.
        \end{step}
        
        \end{example}
    
    \subsection{Yates Correction}
        For contingency tables that are 2x2 we must use yate's correction since it produces a more accurate result.
        
        $$
        \chi^2 = \frac{(|O_i - E_i| - 0.5)^2}{E_i}
        $$
    
        \begin{example}
        {
            Test the results of the following drug trial at a 5\% confidence interval.
            
            \begin{center}
            \begin{tabular}{c|c|c}
                        & Cold  & No cold   \\
                \hline
                Drug    & 32    & 66        \\
                Placebo & 45    & 55        \\
            \end{tabular}
            \end{center}
        }
        
        
        \begin{step}{Hypothesis}
        $H_0$: No association
        
        $H_1$: Association
        \end{step}
        
        \begin{step}{Calculate the expected and $\chi^2$ values}
        \begin{center}
        \begin{tabular}{c|c|c}
            $O_i$ & $E_i$ & $\displaystyle\frac{(|O_i - E_i| - 0.5)^2}{E_i}$    \\[2ex]
            \hline
            \rule{0pt}{3.5ex} 
            34    & 36.5  & 0.633                                               \\[1ex]
            45    & 39.5  & 0.633                                               \\[1ex]
            66    & 60.5  & 0.413                                               \\[1ex]
            55    & 60.5  & 0.413                                               \\[1ex]
        \end{tabular}
        \end{center}
        \end{step}
        
        \begin{step}{Find the statistic $\chi^2$}
        $$\sum{\frac{(|O_i - E_i| - 0.5)^2}{E_i}} = \chi^2 = 2.09$$
        \end{step}
        
        \begin{step}{Find the critical value}
        $$v = (2-1)*(2-1) = 1$$
        $$\chi^2_1(0.95) = 3.841$$
        \end{step}
        
        \begin{step}{Conclusion}
        $2.09 < 3.841$ so accept $H_0$, no evidence to show drug is effective.
        \end{step}
        
        \end{example}
\section{Distribution free methods}
    Distribution free test's are tests that do not require the data to follow a particular distribution. Their are four Distribution free methods you need to know:

    \begin{center}
    \begin{tabulary}{\linewidth}{|L|L|L|}
    \hline
    Test & Test for & When to use \\\hline
    The sign test & Median & Cant use Wilcoxon (data is not symmetrical) \\\hline
    Wilcoxon signed-rank test & Median/mean & Cant use z or t test \\\hline
    Mann-Whitney U test & Two samples have identical populations & Their are two samples \\\hline
    Kruskal-Wallis test & More than two samples from identical populations & Their are more than two samples \\\hline
    \end{tabulary}
    \end{center}

    \subsection{The Sign Test}
        
        The sign test works by checking if their is a significant difference in the median value of the samples by comparing each value pair. Unlike the Wilcoxon test the distribution does not have to be symmetrical.
        
        \subsubsection{Process}
            \begin{enumerate}
            \item Hypothesis.
            \item Find the sign of the difference ignoring any pairs that are equal.
            \item Count the number of positive and negative signs (these are the test statistics)
            \item Find the value of $P(X < min(a, b) | X \sim B(n, \frac{1}{2}))$ where $a$ and $b$ are the test statistics and $n$ is the number of pairs (minus equal pairs)
            \item Compare this value with the significant level, if its less than the significant level reject $H_0$
            \end{enumerate}
        
        \subsubsection{1-way Example}
            \begin{example}
            {
            Results from a 2005 sample of cocaine usage: 9, 26, 17, 18, 21, 16, 19, 13, 15. In 2000 the average was 14, test at the 10\% significance level if cocaine usage has risen.
            }
        
            \begin{step}{Hypothesis}
            $H_0$: Population median $\eta = 14$
            
            $H_1$: Population median $\eta > 14$
            \end{step}
            
            \begin{step}{Signs}
            \begin{center}
            \begin{tabular}{c|c|c|c|c|c|c|c|c|c}
                $x - 14$ & $-5$ & $12$ & $3$  & $4$  & $7$  & $2$  & $5$  & $-1$ & $1$ \\
                \hline
                Signs    & $-$  & $+$  & $+$  & $+$  & $+$  & $+$  & $+$  & $-$  & $+$  
            \end{tabular}
            \end{center}
            \end{step}
            
            \begin{step}{Test Statistic}
            $$2^- / 7^+$$
            \end{step}
            
            \begin{step}{Critical value}
            $$X \sim B(9, \frac{1}{2})$$
            $$P(X \leq 2) = 0.0898$$
            \end{step}
            
            \begin{step}{Conclusion}
            $0.0898 < 0.10$ so reject $H_0$, significant evidence at 10\% level to suggest median cocaine usage has increased since 2000.
            \end{step}
            
            \end{example}
        
        \subsubsection{2-way Example}
            \begin{example}
            {
            Test is their is a difference in the aerosols average effectiveness at a 5\% significance level.
            }
        
            \begin{step}{Hypothesis}
            $H_0$: Population median difference $\eta_d = 0$
            
            $H_1$: Population median difference $\eta_d \neq 0$
            \end{step}
            
            \begin{step}{Signs}
            \begin{center}
            \begin{tabular}{c|c|c|c|c|c|c|c|c|c}
                Patient     & 1   & 2   & 3   & 4   & 5   & 6   & 7   & 8   & 9   \\
                \hline
                Aerosol A   & 28  & 22  & 10  & 40  & 18  & 52  & 49  & 40  & 34  \\
                Aerosol B   & 24  & 16  & 5   & 17  & 23  & 57  & 30  & 16  & 14  \\
                Sign        & $+$ & $+$ & $+$ & $+$ & $-$ & $-$ & $+$ & $+$ & $+$ \\
            \end{tabular}
            \end{center}
            \end{step}
            
            \begin{step}{Test Statistic}
            $$2^- / 7^+$$
            \end{step}
            
            \begin{step}{Critical value}
            $$X \sim B(9, \frac{1}{2})$$
            
            $$P(X \leq 2) = 0.0898$$
            \end{step}
            
            \begin{step}{Conclusion}
            $0.0898 > 0.05$ so accept $H_0$, significant evidence at 5\% level to suggest their is no difference in the average effectiveness of aerosols.
            \end{step}
            
            \end{example}
        
    \subsection{Wilcoxon Signed-rank Test}
    
        Wilcoxon signed-rank test is also a distribution free test, for when we can't use a z or t test. Its similar to the sign test except we rank the differences \(ignoring the signs, smallest first\), then add the ranks with the same signs.
        
        \subsubsection{Process}
            \begin{enumerate}
                \item Hypothesis
                \item Rank the absolutes of the differences.
                \item Give the ranks the same sign as the difference.
                \item Sum the positive then the negative ranks and pick the smallest value.
                \item Find the critical value in the table.
                \item If the statistic is less than the critical value reject $H_0$.
            \end{enumerate}
            
        \begin{example}
        {
            Test if the distributions of mock and a-level results are the same.
        }
            
        \begin{step}{Hypothesis}
        $H_0$: Population median difference $\eta_d = 0$
        
        $H_1$: Population median difference $\eta_d > 0$
        \end{step}
        
        \begin{step}{Find the difference, rank and signed rank}
        \begin{center}
        \begin{tabular}{c|c|c|c|c|c}
            Candidate & Mock & A level & Difference & Rank & Signed rank \\
            1         & 40   & 45      & 5          & 7    & 7           \\
            2         & 65   & 68      & 3          & 4    & 4           \\
            3         & 53   & 47      & -6         & 9.5  & -9.5        \\
            4         & 79   & 75      & -4         & 6    & -6          \\
            5         & 87   & 88      & 1          & 1    & 1           \\
            6         & 70   & 88      & 18         & 13   & 13          \\
            7         & 80   & 77      & -3         & 4    & -4          \\
            8         & 63   & 69      & 6          & 9.5  & 9.5         \\
            9         & 51   & 60      & 9          & 12   & 12          \\
            10        & 82   & 88      & 6          & 9.5  & 9.5         \\
            11        & 27   & 30      & 3          & 4    & 4           \\
            12        & 71   & 73      & 2          & 2    & 2           \\
            13        & 29   & 35      & 6          & 9.5  & 9.5         \\
        \end{tabular}
        \end{center}
        \end{step}
        
        \begin{step}{Test statistics}
        \begin{align*}
        T_+ &= \sum{+ \text{Rank}} = 71.5\\
        T_- &= \sum{- \text{Rank}} = 19.5\\
        T &= 19.5\\
        \end{align*}
        \end{step}
        
        \begin{step}{Critical value}
        $$\text{Critical value} = 21$$
        \end{step}
        
        \begin{step}{Conclusion}
        $19.5 < 21$ so reject $H_0$, evidence students did better in their A-levels.
        \end{step}
        \end{example}
        
    \subsection{Mann-Whitney U-Test}
    
        Mann-Whitney U-Test is another non-parametric test that is used to test if two sample were taken from the same population. It is used when we can do a t-test because the data is not normal.
        
        \subsubsection{Process}
            \begin{enumerate}
                \item Hypothesis
                \item Rank the data (as if it was just a single set) and calculate the sum of the ranks of each set.
                \item Calculate the statistic for each set using $U = T - \displaystyle\frac{n(n+1)}{2}$ where $T$ is the sum of the ranks and $n$ is the sample size. The test statistic is the minimum of the two $U$ values.
                \item Find the critical value in the table.
                \item If the statistic is less than the critical value reject $H_0$.
            \end{enumerate}
   
        \begin{example}
        {
        Nine random plants from two sides of a valley where weighed, carry out a Mann-Whitney U test at 5\% level of significance.
        
        \begin{center}
        \begin{tabular}{c|c|c|c|c|c|c|c|c|c}
            East Side & $$27.1$$ & $$40.3$$ & $$15.7$$ & $$36.4$$ & $$16.3$$ & $$15.3$$ & $$32.0$$ & $$15.7$$ & $$27.5$$ \\
            \hline
            West Side & $$11.7$$ & $$14.7$$ & $$19.1$$ & $$22.0$$ & $$6.7$$ & $$14.1$$ & $$20.1$$ & $$24.4$$ & $$15.4$$
        \end{tabular}
        \end{center}
        }
        
        \begin{step}{Hypothesis}
        $H_0$: Samples taken from identical populations
        
        $H_1$: Samples not taken from identical populations
        \end{step}
        
        \begin{step}{Rank the data}
        \begin{center}
        \begin{tabular}{c|c|c|c|c|c|c|c|c|c}
        East Side & $14$ & $18$ & $7.5$ & $17$ & $9$ & $5$ & $16$ & $7.5$ & $15$  \\
        \hline
        West side & $2$  & $4$  & $10$  & $12$ & $1$ & $3$ & $11$ & $13$  & $6$   \\
        \end{tabular}
        \end{center}
        
        \begin{align*}
        T_E &= 14 + 18 + ... + 15 = 109\\
        T_W &= 2 + 5 + ... + 6 = 62
        \end{align*}
        \end{step}

        \begin{step}{Test Statistic}
        \begin{align*}
        U_E &= 109 - \frac{9 \times (9 + 1)}{2} = 64\\
        U_W &= 62 - \frac{9 \times (9 + 1)}{2} = 17\\
        U &= min(64, 17) = 17
        \end{align*}
        \end{step}

        \begin{step}{Critical value}
        $$\text{Critical value} = 18$$
        \end{step}

        \begin{step}{Conclusion}
        $17 < 18$ so reject $H_0$, the population average weight differs across each side.
        \end{step}
        \end{example}
    
    \subsection{Kruskal-Wallis Test}
        Kruskal-Wallis test is a non-parametric version of an ANOVA test. The test determins if their is a difference between some samples. We use this test instead of an ANOVA test if the data is not normally distributed.

        \subsubsection{Process}
            \begin{enumerate}
            \item Hypothesis
            \item Rank the data (as if it was a single set) and calculate the sum of the ranks of each set.
            \item Calculate the test statistic using: $H = \displaystyle\frac{12}{N(N + 1)} \times \displaystyle\sum{\frac{T_i^2}{n_i}} - 3(N + 1)$ where $T_i$ is the sum of all ranks in a set of size $n_i$ and $N$ is the sum of all sample sizes.
            \item Find the degrees of freedom, which is the number of samples minus one.
            \item Use the $\chi^2$ distribution to find the critical value.
            \item Compare the test statistic and critical value, if statistic is larger than critical value reject $H_0$.
            \end{enumerate}

        \begin{example}
        {
            Carry out a Kruskal-Wallis test to see if their is a difference between the fish prices from three different markets at 5\% significance level.

            \begin{center}
            \begin{tabular}{c|c|c}
                A     & B     & C     \\
                \hline
                220.3 & 190.1 & 228.7 \\
                226.3 & 209.7 & 231.3 \\
                227.3 & 223.4 & 249.6 \\
                228.1 & 224.2 & 260.7 \\
                242.4 & 226.4 & 289.7
            \end{tabular}
            \end{center}
        }

        \begin{step}{Hypothesis}
        $H_0$: Samples from the same population
        
        $H_1$: Samples from the different populations
        \end{step}

        \begin{step}{Rank the data}
        \begin{center}
        \begin{tabular}{c|c|c}
        A  & B & C  \\
        \hline
        3  & 1 & 10 \\
        6  & 2 & 11 \\
        8  & 4 & 13 \\
        9  & 5 & 14 \\
        12 & 7 & 15
        \end{tabular}
        \end{center}
        \end{step}

        \begin{step}{Test statistic}
        \begin{align*}
        T_A &= 3 + 6 + ... + 12 = 38\\
        T_B &= 1 + 2 + ... + 7 = 19\\
        T_C &= 10 + 11 + ... + 15 = 63\\
        \end{align*}

        \begin{align*}
        \sum\frac{T_i}{n_i} &= \frac{38^2}{5} + \frac{19^2}{5} + \frac{63^2}{5}\\
        &= 1154.8
        \end{align*}

        \begin{align*}
        H &= \frac{12}{15(15 + 1)} \times 1154.8 - 3(15 + 1)\\
        &= 9.74
        \end{align*}
        \end{step}

        \begin{step}{Degrees of freedom}
        \begin{align*}
        v &= 3 - 1 \\
          &= 2
        \end{align*}
        \end{step}

        \begin{step}{Critical value}
        $$
        \chi^2_2(5\%) = 5.991
        $$
        \end{step}
        
        \begin{step}{Conclusion}
        $9.74 > 5.991$ so reject $H_0$, samples are not from identical populations
        \end{step}
        \end{example}
\section{Correlation}
    \subsection{Spearman's Rank Correlation Coefficent}

        The product moment correlation coefficient works when you have something to measure. Suppose you had an order, for example an order of preference in different blends of tea, this is when we use Spearman's rank correlation coefficient.

        To calculate the Spearman's rank correlation coefficient we use the following:
        $$
        r_s = 1 - \frac{6\sum{d^2}}{n(n^2-1)}
        $$
        Where $d_i$ is the difference between rank $x_i$ and rank $y_i$ and $n$, the number of pairs.

        \begin{example}
        {
            Two judges judge a competition, comment on how well they agree

            \begin{center}
            \begin{tabular}{l|c|c|c|c|c|c|c|c|c|c}
            Competitor & A & B & C & D & E & F & G & H & I & J \\
            \hline
            Judge 1 & $7.8$ & $6.6$ & $7.3$ & $7.4$ & $8.4$ & $6.5$ & $8.9$ & $8.5$ & $6.7$ & $7.7$ \\
            Judge 2 & $8.1$ & $6.8$ & $8.2$ & $7.5$ & $8.0$ & $6.7$ & $8.5$ & $8.3$ & $6.6$ & $7.8$ \\
            \end{tabular}
            \end{center}
        }

        \begin{step}{Rank the scores and find the difference}
        \begin{center}
        \begin{tabular}{l|c|c|c|c|c|c|c|c|c|c}
        Competitor & A & B & C & D & E & F & G & H & I & J \\
        \hline
        $x_i$ & 4 & 9 & 7 & 6 & 3 & 10 & 1 & 2 & 8 & 5 \\
        $y_i$ & 4 & 8 & 3 & 7 & 5 & 9 & 1 & 2 & 10 & 6 \\
        $d_i$ & 0 & 1 & 4 & -1 & -2 & 1 & 0 & 0 & -2 & -1 \\
        \end{tabular}
        \end{center}
        \end{step}

        \begin{step}{Calculate to coefficient}
        \begin{align*}
        r_s &= 1 - \frac{6 \times (0^2 + 2^2 +4^2 + ...)}{10 \times (10^2-1)}\\
        &= 1 - \frac{168}{990}\\
        &= 0.830
        \end{align*}
        \end{step}

        \begin{step}{Conclusion}
        Fairly strong correlation between the judges, suggesting judges have similar criteria and standards.
        \end{step}

        \end{example}

    \subsection{Testing the correlation coefficient}
        We may be interested in conducting a test to show their is no correlation between two random variables. First we find the correlation coefficient, then get the critical value from the appropriate table.

        \begin{example}
        {
            Test to see if their is no correlation between English and maths marks at the 5\% significance level.

            \begin{center}
            \begin{tabular}{l|c|c|c|c|c|c|c|c}
            Student        & A  & B  & C  & D  & E  & F  & G  & H  \\
            English        & 25 & 18 & 32 & 27 & 21 & 35 & 28 & 30 \\
            Mathematics    & 16 & 11 & 20 & 17 & 15 & 26 & 32 & 20 \\
            \end{tabular}
            \end{center}
        }

        \begin{step}{Hypothesis} 
        $$H_0: \rho = 0$$

        $$H_1: \rho \ne 0$$
        \end{step}

        \begin{step}{Calculate the product moment correlation coefficient}
        \begin{align*}
        \sum{x_i} &= 216\\
        \sum{y_i} &= 157\\
        \sum{x_i^2} &= 6052\\
        \sum{y_i^2} &= 3391\\
        \sum{x_i y_i} &= 4418\\
        S_{xx} &= 6052 - \frac{216^2}{8} = 220\\
        S_{yy} &= 3391 - \frac{157^2}{8} = 309.875\\
        S_{xy} &= 4418 - \frac{216 \times 157}{8} = 179\\
        r &= \frac{179}{\sqrt{220 \times 309.875}} = 0.686
        \end{align*}
        \end{step}

        \begin{step}{Critical value (from table)}
        $$
        \text{critical value} = \pm 0.7067
        $$
        \end{step}

        \begin{step}{Conclusion}
        $-0.7067 < 0.686 < 0.7067$ so value is not in critical region, accept $H_0$, evidence to suggest product moment correlation coefficient is zero.
        \end{step}

        \end{example}